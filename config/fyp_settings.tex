% fyp_proposal_settings.tex
% -------------------------------------------------------------------
%   XJTLU FYP Proposal - Custom Configuration for FAST Template
% -------------------------------------------------------------------

% === 1. 字体和行距设置 ===
% 加载 Times New Roman 字体宏包
\usepackage{newtxtext, newtxmath}
% 设置全局行距为 1.5 倍
\onehalfspacing
% 两端对齐
\usepackage{ragged2e}

% === 2. 章节标题格式化 (核心要求) ===
% 使用 titlesec 宏包重新定义章节标题样式
% \section 对应一级标题 (Introduction), 设置为 20pt
\titleformat{\section}
    {\normalfont\bfseries\fontsize{20}{24}\selectfont} % 格式
    {\thesection} % 标签
    {1em} % 标签与标题之间的距离
    {} % 标题前的代码

% \subsection 对应二级标题 (Introduction, Background), 设置为 16pt
\titleformat{\subsection}
    {\normalfont\bfseries\fontsize{16}{19}\selectfont}
    {\thesubsection}
    {1em}
    {}

% \subsubsection 对应三级标题 (Motivation, Objectives), 设置为 14pt
\titleformat{\subsubsection}
    {\normalfont\bfseries\fontsize{14}{17}\selectfont}
    {\thesubsubsection}
    {1em}
    {}

% 确保标题后的第一段也进行首行缩进
\usepackage{indentfirst}

\providecommand{\TemplateSupervisorName}{}

% === 3. 封面页定制 ===
% 移除 FAST 模板默认的 \maketitle 命令,我们将手动构建封面
% \renewcommand{\maketitle}{}

% 创建一个新的命令来生成符合 FYP 要求的封面
\renewcommand{\maketitle}{
    \begin{titlepage}
        \centering
        % XJTLU Logo
        \includegraphics[width=0.8\textwidth]{assets/images/xjtlu-title.png}\par
        \vspace{2cm}
        {\huge\bfseries \TemplateModuleCode~\TemplateModuleName}\par
        \vspace{2cm}
        {\Huge\bfseries \TemplateModuleName}\par % 报告标题
        \vspace{1cm}
        {\LARGE Proposal Report}\par
        \vfill
        {\large In Partial Fulfillment}\par
        {\large of the Requirements for the Degree of}\par
        {\large Bachelor of Engineering}\par
        \vspace{2cm}
        % 学生信息
        \large
        \begin{tabular}{|ll|p{0.4\textwidth}|}
            \hline
            Student Name & : & \TemplateAuthorName \\
            \hline
            Student ID   & : & \TemplateAuthorID \\
            \hline
            Supervisor   & : & \TemplateSupervisorName \\
            \hline
        \end{tabular}\par
        \vfill
        {\large School of AI and Advanced Computing}\par
        {\large Xi’an Jiaotong-Liverpool University}\par
        November 2025
    \end{titlepage}
}

% === 4. 页眉页脚调整 (可选) ===
% 如果需要,可以在这里调整页眉页脚以符合特定要求
% 示例:移除页眉,仅保留页脚的页码和学生信息
\pagestyle{fancy}
\fancyhf{} % 清空所有页眉页脚
\fancyfoot[C]{\thepage} % 中间页码
\renewcommand{\headrulewidth}{0pt} % 移除页眉横线
\renewcommand{\footrulewidth}{0pt} % 移除页脚横线

% === 5. 附录格式设置(仅在 \appendix 之后生效) ===
\usepackage[toc,page]{appendix}
\usepackage{titlesec}

% 保存正文 section/subsection 样式
\let\origsection\section
\let\origsubsection\subsection

% 定义一个钩子:当进入附录时替换标题格式
\AtBeginEnvironment{appendices}{
  % 修改标题格式
  \titleformat{\section}
    {\normalfont\bfseries\fontsize{18}{22}\selectfont}
    {Appendix~\Alph{section}.}
    {1em}{}
  \titleformat{\subsection}
    {\normalfont\bfseries\fontsize{16}{19}\selectfont}
    {\Alph{section}.\arabic{subsection}}
    {1em}{}

  % 修改目录中的标题
  \renewcommand{\appendixtocname}{Appendices}
  \renewcommand{\appendixpagename}{Appendices}
}