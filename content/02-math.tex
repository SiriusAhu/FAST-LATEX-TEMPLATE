\section{Mathematics}

Look, {\LaTeX} supports math expressions by default!
You can use inline math like \( e^{i\pi} + 1 = 0 \) or \( a^2 + b^2 = c^2 \) within \texttt{\textbackslash(...\textbackslash)} or \texttt{\$...\$}.

Or display full equations on their own line by using \texttt{\textbackslash[...\textbackslash]} or classically, using \texttt{\$\$...\$\$} like these:

\[
  \nabla \cdot \mathbf{E} = \frac{\rho}{\epsilon_0}
\]

$$
  \int_{a}^{b} f(x) \, \mathrm{d}x = F(b) - F(a)
$$

If you want to number equations for referencing, use the \texttt{equation} environment:

\begin{equation}
  \label{eq:newton}
  \sum_{i=1}^{n} \vec{F}_i = m \cdot \vec{a}
\end{equation}

For multi-line derivations, use the \texttt{align} environment:
\begin{align}
  E &= mc^2, \\
  \frac{\mathrm{d}}{\mathrm{d}t} \int_{V} \rho \, \mathrm{d}V &= - \int_{S} \rho \vec{v} \cdot \mathrm{d}\vec{S}.
\end{align}
