\newpage
\section{Figures}

Use the \texttt{graphicx} package to insert figures (images). Store assets under \texttt{assets/images/} to keep the project tidy.

2 customized commands are provided for convenience. 

First, \texttt{\textbackslash picHere} wraps a full \texttt{figure} environment with caption and label support. (Figure \ref{fig:example-figure} is an example.)

The basic usage pattern is:

\begin{itemize}
  \item \texttt{\textbackslash picHere\{path\}\{width\}\{caption\}\{label\}} inserts a centred figure with a caption and a cleveref-compatible label.
\end{itemize}

Here is an example of how to use it:

\begin{lstlisting}[style=githubLight, language={[LaTeX]TeX}, caption={Using the figure helpers in a listing block.}, label={lst:figure-helper-macros}]
\picHere{assets/images/github-icon.png}{0.7\textwidth}{Example figure included from external asset.}{fig:example-figure}
\end{lstlisting}

\picHere{assets/images/github-icon.png}{0.7\textwidth}{Example figure included from external asset.}{fig:example-figure}

While \texttt{\textbackslash picHereSimple} is a minimal drop-in for decorative images that do not require referencing. (No caption or label, just the image.)

The usage pattern is:

\begin{itemize}
  \item \texttt{\textbackslash picHereSimple\{path\}\{width\}} gives you the same layout without caption or label when the image is purely decorative.
\end{itemize}

Here is an example of how to use it:

\begin{lstlisting}[style=githubLight, language={[LaTeX]TeX}, caption={Using the figure helpers in a listing block.}, label={lst:figure-helper-macros}]
    \picHereSimple{assets/images/github-icon.png}{0.7\textwidth}
\end{lstlisting}

\picHereSimple{assets/images/github-icon.png}{0.7\textwidth}