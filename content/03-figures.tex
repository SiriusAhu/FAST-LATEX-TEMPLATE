\newpage
\section{Figures}

Figure insertion is another basic and essential feature of {\LaTeX}.However, the default syntax is quite verbose and not very user-friendly.

That's why \textbf{FAST} introduces some custom commands to simplify the process!

\noindent 2 customized commands are provided for convenience. 

\subsection{\texttt{\textbackslash picHere}: Full-featured figure insertion}

\texttt{\textbackslash picHere} wraps a full \texttt{figure} environment with caption and label support (4 arguments are required). (Figure \ref{fig:example-figure} is an example.)

\noindent The following snippet and its output demonstrate how to use it:

\begin{lstlisting}[style=githubLight, language={[LaTeX]TeX}, label={lst:figure-helper-macros}]
\picHere{assets/images/github-icon.png}{0.8\textwidth}{Example figure included from external asset.}{fig:example-figure}
\end{lstlisting}

\picHere{assets/images/github-icon.png}{0.8\textwidth}{Example figure included from external asset.}{fig:example-figure}

\subsection{\texttt{\textbackslash picHereSimple}: Minimal decorative figure insertion}

While \texttt{\textbackslash picHereSimple} is a minimal drop-in for decorative images that do not require referencing. (No caption or label, just the image.)

\noindent The following snippet and its output demonstrate how to use it:

\begin{lstlisting}[style=githubLight, language={[LaTeX]TeX}, label={lst:figure-helper-macros}]
    \picHereSimple{assets/images/github-icon.png}{0.8\textwidth}
\end{lstlisting}

\picHereSimple{assets/images/github-icon.png}{0.8\textwidth}